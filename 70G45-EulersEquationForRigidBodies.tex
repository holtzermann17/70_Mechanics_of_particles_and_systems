\documentclass[12pt]{article}
\usepackage{pmmeta}
\pmcanonicalname{EulersEquationForRigidBodies}
\pmcreated{2013-03-22 17:10:36}
\pmmodified{2013-03-22 17:10:36}
\pmowner{perucho}{2192}
\pmmodifier{perucho}{2192}
\pmtitle{Euler's equation for rigid bodies}
\pmrecord{8}{39490}
\pmprivacy{1}
\pmauthor{perucho}{2192}
\pmtype{Topic}
\pmcomment{trigger rebuild}
\pmclassification{msc}{70G45}

\endmetadata

% this is the default PlanetMath preamble.  as your knowledge
% of TeX increases, you will probably want to edit this, but
% it should be fine as is for beginners.

% almost certainly you want these
\usepackage{amssymb}
\usepackage{amsmath}
\usepackage{amsfonts}

% used for TeXing text within eps files
%\usepackage{psfrag}
% need this for including graphics (\includegraphics)
%\usepackage{graphicx}
% for neatly defining theorems and propositions
%\usepackage{amsthm}
% making logically defined graphics
%%%\usepackage{xypic}

% there are many more packages, add them here as you need them

% define commands here
\newtheorem{theorem}{Theorem}
\newtheorem{defn}{Definition}
\newtheorem{prop}{Proposition}
\newtheorem{lemma}{Lemma}
\newtheorem{cor}{Corollary}

\begin{document}
Let $1$ be an inertial frame body (a rigid body) and $2$ a rigid body in motion respect to an observer located at $1$. Let $Q$ be an arbitrary point (fixed or in motion) and $C$ the center of mass of $2$. Then,
\begin{align}
\mathbf{M}_Q=\boldsymbol{\mathbb{I}}^Q\boldsymbol{\alpha}_{21}+
\boldsymbol{\omega}_{21}\times(\boldsymbol{\mathbb{I}}^Q\boldsymbol{\omega}_{21})+
m\mathbf{\overline{QC}}\times \mathbf{a}^{Q2}_1,
\end{align}
where $m$ is the mass of the rigid body, $\mathbf{\overline{QC}}$ the position vector of $C$ respect to $Q$, $\mathbf{M}_Q$ is the moment of forces system respect to $Q$, $\boldsymbol{\mathbb{I}}^Q$ the tensor of inertia respect to orthogonal axes embedded in $2$ and origin at $Q2$ {\footnote{That is possible because the kinematical concept of frame extension.}}, and $\mathbf{a}^{Q2}_1$, $\boldsymbol{\omega}_{21}$, $\boldsymbol{\alpha}_{21}$, are the acceleration of $Q2$, the angular velocity and acceleration vectors respectively, all of them measured by an observer located at $1$. \\
This equation was got by Euler by using a  fixed system of principal axes with origin at $C2$. In that case we have $Q=C$, and therefore
\begin{align}
\mathbf{M}_C=\boldsymbol{\mathbb{I}}^C\boldsymbol{\alpha}_{21}+ \boldsymbol{\omega}_{21}\times(\boldsymbol{\mathbb{I}}^C\boldsymbol{\omega}_{21}).
\end{align}
Euler used three independent scalar equations to represent (2). It is well known that the number of degrees of freedom associate to a rigid body in free motion in $\mathbb{R}^3$ are six, just equal the number of independent scalar equations necessary to solve such a motion. (Newton's law contributing with three) \\
Its is clear if $2$ is at rest or in uniform and rectilinear translation, then $\mathbf{M}_Q=\mathbf{0}$, one  of the necessary and sufficient conditions for the equilibrium of the system of forces applied to a rigid body. (The other one is  the force resultant $\mathbf{F}=\mathbf{0}$)




%%%%%
%%%%%
\end{document}
