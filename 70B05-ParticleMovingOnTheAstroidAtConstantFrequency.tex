\documentclass[12pt]{article}
\usepackage{pmmeta}
\pmcanonicalname{ParticleMovingOnTheAstroidAtConstantFrequency}
\pmcreated{2013-03-22 17:14:09}
\pmmodified{2013-03-22 17:14:09}
\pmowner{perucho}{2192}
\pmmodifier{perucho}{2192}
\pmtitle{particle moving on the astroid  at constant frequency}
\pmrecord{9}{39565}
\pmprivacy{1}
\pmauthor{perucho}{2192}
\pmtype{Topic}
\pmcomment{trigger rebuild}
\pmclassification{msc}{70B05}

% this is the default PlanetMath preamble.  as your knowledge
% of TeX increases, you will probably want to edit this, but
% it should be fine as is for beginners.

% almost certainly you want these
\usepackage{amssymb}
\usepackage{amsmath}
\usepackage{amsfonts}

% used for TeXing text within eps files
%\usepackage{psfrag}
% need this for including graphics (\includegraphics)
%\usepackage{graphicx}
% for neatly defining theorems and propositions
%\usepackage{amsthm}
% making logically defined graphics
%%%\usepackage{xypic}

% there are many more packages, add them here as you need them

% define commands here
\newtheorem{theorem}{Theorem}
\newtheorem{defn}{Definition}
\newtheorem{prop}{Proposition}
\newtheorem{lemma}{Lemma}
\newtheorem{cor}{Corollary}

\begin{document}
In parametric Cartesian equations, the astroid can be represented by
$$x = a\cos^3\omega t,\quad y = a\sin^3\omega t,$$
where $a>0$ is a known constant, $\omega>0$ is the constant angular frequency, and $t\in [0,\infty)$ is the time parameter. Thus the position vector of a particle, moving over the astroid, is
$$\mathbf{r}=a\cos^3\omega t\,\mathbf{i}+a\sin^3\omega t\,\mathbf{j},$$
and its velocity
$$\mathbf{v}=-3a\omega\sin\omega t\cos^2\omega t\,\mathbf{i}+3a\omega\sin^2\omega t\cos\omega t\,\mathbf{j},$$
where $\{\mathbf{i},\mathbf{j}\}$ is a reference basis. Hence for the particle speed we have
$$v=3a\omega\sin\omega t\cos\omega t.$$
From the last two equations we get the tangent vector 
$$\mathbf{T}=-\sin\omega t\,\mathbf{i}+\cos\omega t\,\mathbf{j},$$
and by using the well known formula {\footnote{By applying the chain rule,
$$\bigg\Vert\frac{d\mathbf{T}}{dt}\bigg\Vert=\bigg\Vert\frac{d\mathbf{T}}{ds}\bigg\Vert\bigg\vert\frac{ds}{dt}\bigg\vert=
\bigg\Vert\frac{\mathbf{N}}{\rho}\bigg\Vert v=\frac{v}{\rho},$$
by Frenet-Serret. $\mathbf{N}$ is the normal vector.}}
$$\bigg\Vert\frac{d\mathbf{T}}{dt}\bigg\Vert=\frac{v}{\rho},$$
$\rho>0$ being the radius of curvature at any instant $t$, we arrive to the useful equation
$$v=\omega\rho.$$

%%%%%
%%%%%
\end{document}
